% Contains user definitions %%%%%%%%%%%%%%%%%%%%%%%%%%%%%%%%%%%%%%%%%%%%%%%%%%%%%%%%%%%%%%%%%%%%%%%%%%%%%%%%%%%%%%%%%%%%%% 
\makeatletter
\newcommand{\inlineparagraph}{\@startsection{paragraph}{4}{\z@}%
  {-3.25ex \@plus -1ex \@minus -0.2ex}%
  {-1.2pt}%
  {}%
}

\newcommand{\noskipparagraph}{\@startsection{paragraph}{4}{\z@}%
  {-3.25ex \@plus -1ex \@minus -0.2ex}%
  {0.0001px}%
  {\scshape}%
}

\newcommand{\smallskipparagraph}{\@startsection{paragraph}{4}{\z@}%
  {-3.25ex \@plus -1ex \@minus -0.2ex}%
  {5px}%
  {\scshape}%
}
\makeatother

\newcommand{\ilparagraph}[1]{\inlineparagraph{\bfseries #1\textcolor{cern}{.}}}
% \newcommand{\myparagraph}[1]{\noskipparagraph{#1}}
% \newcommand{\myparagraph}[1]{\smallskipparagraph{\bfseries #1}}

% As the paragraphs don't have numbers, cref can't 
% refer to them. This makes cref use their name instead:
% \crefname{page}{page}{page}  % probably not needed
\crefformat{myparagraphlink}{#2\textsc{\textbf{#1}}#3}

\newcounter{myparagraphlink}
\makeatletter
\newcommand{\myparagraph}[1]{%
  \smallskipparagraph{\bfseries #1}%
  \refstepcounter{myparagraphlink}%
  \def\cref@currentlabel{[myparagraphlink][\arabic{myparagraphlink}][]#1}%
  \def\@currentlabelname{#1}%
}
\makeatother

% lengths
\newlength{\figurewidth}
\setlength{\figurewidth}{\linewidth}

\newlength{\subfigurewidth}
\setlength{\subfigurewidth}{0.5\textwidth}

% text arrangement
\newcommand{\tstack}[2]{$\substack{\text{\normalsize #1}\\\text{\tiny #2}}$} % stack two lines

% borders
\newcommand{\mytopmargin}[1]{\newgeometry{top=3cm}#1\restoregeometry}
\newcommand{\tinytopmargin}[1]{\newgeometry{top=1cm}#1\restoregeometry}

\newcommand{\tmpcapskip}[1]{
    \setlength{\abovecaptionskip}{5pt}
    \setlength{\belowcaptionskip}{5pt}
    #1
    \setlength{\abovecaptionskip}{10pt}
    \setlength{\belowcaptionskip}{0pt}
}



% Flowcharts
\usetikzlibrary{arrows, shapes, calc, positioning}
\tikzstyle{base} = [text=white, text centered, minimum width=1em, minimum height=1em]
\tikzstyle{startstop} = [base, ellipse, fill=CernRed!90, draw=CernRed!200]
\tikzstyle{io} = [base, trapezium, trapezium left angle=70, trapezium right angle=110, fill=CernNiceBlue, draw=CernBlue!200]
\tikzstyle{process} = [base, rectangle, rounded corners, fill=AtlasOrange!90, draw=AtlasOrange!200]
% \tikzstyle{process} = [base, rectangle, rounded corners, fill=CernYellow!90, draw=CernYellow!200]
\tikzstyle{test} = [base, signal, signal to=east and west, fill=GreenBellPepper!90, draw=GreenBellPepper!200]
% \tikzstyle{test} = [base, signal, signal to=east and west, fill=CernLightGreen!90, draw=CernLightGreen!200]
\tikzstyle{arrow} = [->, >=latex]

% Other
\newcommand\drawRect[5]{%
    \def\top{#1}
    \def\left{#2}
    \def\right{#3}
    \def\bottom{#4}
    \begin{tikzpicture}[remember picture, overlay]
        \draw[#5, thick] (\left,\top) -- (\right,\top) -- (\right,\bottom) -- (\left,\bottom) -- cycle;    
    \end{tikzpicture}
}

% Own commands and environments ---

% labels


% shortcuts
\newcommand{\ttt}[1]{\texttt{#1}}

\newcommand{\todo}[1]{{\color[RGB]{190, 130, 130} \textbf{TODO }\textit{#1}}}
\newcommand{\bone}{\textcolor{blue}{Beam 1}}
\newcommand{\btwo}{\textcolor{red}{Beam 2}}
\newcommand{\tcern}[1]{\textcolor{cern}{#1}}
\newcommand{\bfcern}[1]{\tcern{\bf#1}}
\newcommand{\beamone}{\textcolor{blue}{Beam 1}}
\newcommand{\beamtwo}{\textcolor{red}{Beam 2}}

\newcommand{\pro}[1]{\\\hspace{1em}{\color{GreenBellPepper}\textbf{+} #1}}
\newcommand{\con}[1]{\\\hspace{1em}{\color{CernRed}\hspace{.1em}\textbf{--}\hspace{.2em} #1}}
\newcommand{\sep}{\\\hspace{1em}{- - -}}

\newcommand{\textAD}[2]{$\partial$Q$_{#1}$/$\partial$2J$_{#2}$}

% maths



\newcommand{\of}[1]{\left(#1\right)}    % 'function of', i.e. (x) with proper parenthesis
\newcommand{\ReOf}[1]{\Re\left[#1\right]}    % 'function of', i.e. (x) with proper parenthesis
\newcommand{\ImOf}[1]{\Im\left[#1\right]}    % 'function of', i.e. (x) with proper parenthesis
\newcommand{\E}[1]{\cdot 10^{#1}}

\newcommand{\fof}[1]{\left(#1\right)}

\newcommand{\overeq}[1]{\overset{\mathmakebox[\widthof{=}]{#1}}{=}}
\newcommand{\myover}[2]{\overset{\mathmakebox[\widthof{#2}]{#1}}{#2}}
\newcommand{\myunder}[2]{\underset{\mathmakebox[\widthof{#2}]{#1}}{#2}}
\newcommand{\myoverunder}[3]{\overset{\mathmakebox[\widthof{#2}]{#1}}{\underset{\mathmakebox[\widthof{#2}]{#3}}{#2}}}

% Environments

\newenvironment{wider}[1][10pt]{
    \begin{columns}
        \column{\dimexpr\paperwidth-#1}
    }{
      \end{columns}%
}

\newlist{mytemize}{itemize}{1}
\setlist[mytemize,1]{topsep=2pt, itemsep=-0.5ex, partopsep=1ex, parsep=1ex, label=\smallbullet, rightmargin=\leftmargin}


\newcommand{\textimportant}[1]{\textbf{\textsc{#1}}}
\newlist{important}{description}{1}
\setlist[important,1]{itemindent=-10pt, leftmargin=10pt-\itemindent, parsep=2pt, font=\textimportant, rightmargin=\leftmargin}


% Underlines https://tex.stackexchange.com/a/27260/213738
\newcommand{\udot}[1]{%
    \tikz[baseline=(todotted.base)]{
        \node[inner sep=1pt,outer sep=0pt] (todotted) {#1};
        \draw[dotted] (todotted.south west) -- (todotted.south east);
    }%
}%

\newcommand{\udensdot}[1]{%
    \tikz[baseline=(todotted.base)]{
        \node[inner sep=1pt,outer sep=0pt] (todotted) {#1};
        \draw[densely dotted] (todotted.south west) -- (todotted.south east);
    }%
}%

\newcommand{\uloosdot}[1]{%
    \tikz[baseline=(todotted.base)]{
        \node[inner sep=1pt,outer sep=0pt] (todotted) {#1};
        \draw[loosely dotted] (todotted.south west) -- (todotted.south east);
    }%
}%

\newcommand{\udash}[1]{%
    \tikz[baseline=(todotted.base)]{
        \node[inner sep=1pt,outer sep=0pt] (todotted) {#1};
        \draw[dashed] (todotted.south west) -- (todotted.south east);
    }%
}%

\newcommand{\udensdash}[1]{%
    \tikz[baseline=(todotted.base)]{
        \node[inner sep=1pt,outer sep=0pt] (todotted) {#1};
        \draw[densely dashed] (todotted.south west) -- (todotted.south east);
    }%
}%

\newcommand{\uloosdash}[1]{%
    \tikz[baseline=(todotted.base)]{
        \node[inner sep=1pt,outer sep=0pt] (todotted) {#1};
        \draw[loosely dashed] (todotted.south west) -- (todotted.south east);
    }%
}%


% Create linkable options
\newcommand{\raisedtaget}[1]{\raisebox{\ht\strutbox}{\hypertarget{#1}{}}}

% replace \_ with _ and assign to \temp.
% this way options with underscores can be used
\newcommand{\subforlink}[1]{%
  \saveexpandmode\noexpandarg
  \StrSubstitute{#1}{\_}{_}[\tttemp]%
  \StrSubstitute{\tttemp}{-}{_}[\temp]%
  \restoreexpandmode
}

% make all options linkable via \opt command below. raisebox is needed, as target is at the base of the line
\newcommand{\defopt}[1]{\subforlink{#1}\raisebox{\ht\strutbox}{\hypertarget{\temp}{}}\texttt{#1}}
\newcommand{\opttarget}[2]{\subforlink{#1}\hypersetup{hidelinks}{\hyperlink{\temp}{\dashuline{#2}}}}
\newcommand{\opt}[1]{\opttarget{#1}{\texttt{#1}}}
\newcommand{\textttu}[1]{\defopt{#1}\texttt{:}}
\newlist{options}{description}{1}
\setlist[options,1]{itemindent=-20pt, leftmargin=20pt-\itemindent, parsep=2pt, font=\textttu, rightmargin=\leftmargin}
% \newlist{options}{itemize}{1}
% \setlist[options,1]{itemindent=0pt, leftmargin=80pt, parsep=2pt, font=\texttt}

% \newcommand{\subdescriptionitem}[1]{\normalfont\textit{#1:}}
\newcommand{\subdescriptionitem}[1]{\textcolor{gray}{\textsc{\small #1}}}
\setlist[description,2]{itemindent=-20pt, leftmargin=-\itemindent, parsep=2pt, font=\subdescriptionitem, rightmargin=\leftmargin}
 
\newcommand{\subtestlistitem}[1]{\textcolor{gray}{\text{\small #1}}}
\newlist{testlist}{description}{2}
\setlist[testlist,1]{itemindent=-10pt, leftmargin=10pt-\itemindent, parsep=2pt, font=\textimportant, rightmargin=\leftmargin}
\setlist[testlist,2]{itemindent=-20pt, leftmargin=-\itemindent, parsep=2pt, font=\subtestlistitem, rightmargin=\leftmargin}


% \setlength{\abovecaptionskip}{-5em}
% \setlength{\belowcaptionskip}{-5em}

