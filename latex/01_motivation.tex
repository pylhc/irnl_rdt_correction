\section{Motivation}

% General Correction in the IRs ---

The sensitivity of accelerator beam optics to magnetic errors depends directly on the $\beta$-function, 
which is highest in the Insertion Regions (IR) around the Interaction Points (IP) 
with the lowest $\beta^*$ (the value of the $\beta$-function at the location of the IP).

Hence, studies of the possibility of correcting the non-linear magnetic errors in these regions in the Large Hadron Collider (LHC)
have already been of significant importance during its design phase:
It was envisaged to make use of the magnetic measurement data of the LHC 
magnets~\cite{SammutMathematicalFormulationPredict2006,SammutMathematicalFormulationPredict2007,SammutMathematicalFormulationPredict2009} to
simulate the machine in MAD-X~\cite{DeniauMADXUserGuide} and calculate the corrections to be used in the
machine~\cite{BruningFieldQualityIssues2004,BruningDynamicApertureStudies2004,TomasNonlinearCorrectionSchemes2009}.
While these simulation-based corrections produced great results in the arcs~\cite{PerssonChromaticCouplingCorrection2013}
in the IRs discrepancies with corrections from beam-based measurements
were observed~\cite{MacleanFirstMeasurementCorrection2015}.
The sources for these discrepancies are still not fully known.
Apart from the successful arc-corrections, 
simulations have nevertheless been a useful tool for the estimation of linear and non-linear 
effects in the IRs~\cite{MacleanNewApproachLHC2019,MacleanFirstMeasurementCorrection2015,
TomasCERNLargeHadron2010,DillyAmplitudeDetuningMisaligned2020}.
Magnetic-measurement based simulations have since supported 
the continuing endeavour to optimize the LHC machine performance
with beam-based corrections in the IRs~\cite{MacleanCommissioningNonlinearOptics2016,
MacleanNewMethodsMeasurement2017,MacleanNewLHCOptics2017,MacleanDetailedReviewLHC2019,
MacleanNewApproachLHC2019,MacleanProspectsBeambasedStudy2022},
and continue to be an invaluable tool in studying future machine layouts,
e.g. the installation of stronger magnets in the IR and the decrease of $\beta^*$ in operation in the High Luminosity upgrade of the 
LHC~(HL-LHC)~\cite{FartoukhHLLHCAcceleratorPhysics2015,AberleHighLuminosityLargeHadron2020},
which is foreseen to result in even tighter constraints on residual errors. 

At the same time the crossing-angle scheme of the collision optics creates large orbit bumps in the IRs,
leading to feed-down effects, the influence of which have been observed and investigated in the LHC.
For both, LHC and HL-LHC the need for corrections of this feed-down has been established
~\cite{MacleanNewMethodsMeasurement2017,
MacleanDetailedReviewLHC2019, 
MacleanNewApproachLHC2019,
MacleanProspectsBeambasedStudy2022,
MacleanFirstMeasurementCorrection2015,
MacleanNonlinearOpticsCommissioning2016,
MacleanLHCMD21712018, 
BuffatOpticsMeasurementCorrection2022}.

% Introduction of correction scheme ---

To estimate the powering of the corrector magnets, a local correction scheme
based on the Resonance Driving Terms (RDTs) in the IRs has been 
utilized~\cite{BruningDynamicApertureStudies2004}. 
Up to now, the implementation of this scheme calculated the correction based on a single input optics, 
for either Beam~1 or Beam~2, and made use of symmetries between the beams to optimize the correction for both beams. 
Cases will occur in which this symmetry does not hold, e.g. through the introduction of feed-down, or the use of inherently asymmetric optics.
An example for the latter is the flat optics~\cite{FartoukhAchromaticTelescopicSqueezing2013,FartoukhFlatTelescopicOptics2018},
in which $\beta^*$ in the two transversal planes no longer has identical values. 
These optics allow for a more distributed radiation deposition in the LHC magnets as well as an increase 
in luminosity~\cite{FartoukhFlatTelescopicOptics2018}.
Their feasibility has been studied during machine developments in the LHC~\cite{FartoukhFirstHighIntensityBeam2019}
and preliminary analysis regarding their influence on corrections and amplitude detuning has been conducted~\cite{DillyCorrectionAmpDet2018}.
A new and flexible version of the correction principle has been implemented~\cite{OMC-TeamIRNLRDTCorrection}, 
taking up to two optics into account and hence not relying on symmetry assumptions, allowing to target RDTs freely,
as well as including feed-down into the calculations. 
The implementation allows for the feed-down from higher orders to the RDT to be corrected, 
as well as using the feed-down from higher order corrector magnets to correct for lower order errors.
Theoretical background and the implementation details of this algorithm are presented in this report.

% The new correction script has since been used for extensive tracking studies, 
% investigating the influence of feed-down~\cite{DillyCorrectionsFeedDownNonLinear2021},
%  the correctability of asymmetric optics~\cite{DillyCorrectionsAsymmetricNonLinear2021}
% and the feasibility to correct systematic normal decapole errors in the separation and 
% recombination dipoles of the HL-LHC~\cite{DillyFeasibilityCorrectingSystematic2021,DillyCorrenctionsSystematicNormal2022}.
% Further studies are still ongoing and will be published in the near future.